\section{Conclusions and assessment}

\subsection{Conclusions}

\subsubsection{Aspetti positivi}

\paragraph{Pagina di Supporto}

Questa pagina è ben organizzata:
\begin{itemize}
  \item È presente una barra di ricerca che permette di cercare delle 
  risposte a dei dubbi che riguardano i prodotti realizzati dall'azienda;

  \item Una sezione che propone alcune FAQ, già suddivise per prodotto;
  
  \item Delle sezioni specifiche che permettono all'utente di raggiungere 
  gli articoli del \textit{Blog} o dell'\textit{Academy} e i vari link 
  social network.
\end{itemize}
Così facendo, il sito suggerisce all'utente:
\begin{itemize}
  \item Prima di trovare la risposta al suo problema andando a cercare 
  tra le risorse pubblicate nel sito (articoli del \textit{Blog} o 
  dell'\textit{Academy});

  \item Di chiedere l'aiuto della community tramite i vari social network.
\end{itemize}


\paragraph{Strumenti di ricerca per il Glossario}

Questa pagina offre degli ottimi strumenti di ricerca di termini. Se 
l'utente conosce il termine da ricercare, può utilizzare la barra di ricerca; 
invece se ricorda parzialmente il termine, può:
\begin{itemize}
  \item Utilizzare comunque la barra di ricerca, in quanto la pagina 
  illustra tutti i risultati che coincidono con i caratteri inseriti nella 
  barra di ricerca;

  \item Può utilizzare lo strumento posto a destra della barra di ricerca: 
  consiste in un alfabeto, in cui ogni lettera è un link che punta ad una 
  sezione in cui vengono illustrati tutti i termini che iniziano per la 
  lettera selezionata.
\end{itemize}
Questi strumenti permettono anche ad un utente neofita del mondo delle 
criptovalute di esplorare il glossario in modo efficiente.

\subsubsection{Aspetti negativi}

\paragraph{Come aprire un ticket nella pagina di Supporto}

Nonostante gli aspetti positivi descritti precedentemente, questa pagina 
non mette in evidenza il bottone che permette di aprire un ticket di 
assistenza (il bottone può essere localizzato in alto a destra, a 
sinistra del bottone verde \textit{Accedi}, fig. \ref{fig:support-1}). 
Questo rappresenta un problema, in quanto l'utente potrebbe trovarsi 
disorientato e non sapere come contattare l'assistenza.

\paragraph{Mancanza di strumenti di ricerca per il Blog e per l'Academy}

Non ci sono degli strumenti di ricerca nè per il \textit{Blog}, nè per 
l'\textit{Academy}. Questo rappresenta un grande svantaggio per l'utente, 
in quanto:
\begin{itemize}
  \item L'utente deve scorrere le varie pagine della sezione \textit{Blog} 
  o \textit{Academy} per poter trovare l'articolo di interesse per l'utente, 
  oppure, 

  \item L'utente deve ricercare l'articolo tramite un motore di ricerca, 
  il quale lo reindirizzerà nuovamente al sito web.
\end{itemize}
La mancanza di questi strumenti sono molto utili per gli utenti (in 
particolare per gli utenti principianti) che necessitano di trovare 
gli articoli per mezzo di parole chiave. Con l'aggiuta di tali strumenti, 
l'usabilità di queste pagine aumenterebbe considerevolmente.

\paragraph{Il Glossario non è strettamente collegato all'Academy}

La pagina del glossario dovrebbe essere facile da raggiungere da un 
articolo o dalla pagina stessa dell'\textit{Academy}. Negli articoli, 
non sempre sono inseriti dei link che reindirizzano al termine 
corrispondente nel glossario. Pertanto, sarebbe utile aggiungere un 
collegamento, in mod che l'utente possa consultare in qualsiasi momento 
il glossario.

\paragraph{FAQ}

Le FAQ sono suddivise per argomento, il che è un aspetto positivo, in quanto 
se l'utente è stato in grado di localizzare il problema, può andare 
direttamente nella sezione specifica. Tuttavia, ci sono tre principali 
problemi:
\begin{itemize}
  \item Queste pagine sono difficili da raggiungere: queste pagine possono 
  essere raggiunte tramite il footer (fig. \ref{fig:footer});
  
  \item Le varie FAQ non sono raggruppate in una unica pagina: per un utente 
  neofita, è utile avere una pagina di riferimento che lo indirizzi 
  verso la sezione specifica (ad esempio, \textit{Tassazione}). In assenza, 
  di una tale pagina, per l'utente è difficile orientarsi e pertanto 
  potrebbe aprire un ticket di assistenza quando la risposta al suo 
  problema potrebbe essere presente per l'appunto nelle FAQ;

  \item Assenza di strumenti di ricerca: non ci sono degli strumenti che 
  permettono di ricercare delle FAQ a partire da alcuni termini inseriti 
  dall'utente. Questo è un grave problema, in quanto avrebbe facilitato 
  l'usabilità per l'utente.
\end{itemize}

\subsection{Assessment}

Sulla base di tutte le analisi effettuate, dalla homepage ad alcune 
pagine interne, questo sito riesce a fornire i contenuti in modo chiaro. 
Le sezioni all'interno delle pagine sono ben organizzate, ma mancano 
degli strumenti che per gli utenti, in particolar modo per i neofiti delle 
criptovalute, sarebbero fondamentali. Quindi, il voto finale che ritengo 
adeguato per il sito web di Young Platform è \textbf{7/10}. Nella tabella 
seguente, si illustrino i voti assegnati per ogni pagina.

\begin{center}
  \begin{longtable}{|c|c|c|}
    \hline
    & \textbf{Page} & \textbf{Grade} \\
    %\endhead
    \hline
    1 & Homepage & 8 \\
    \hline
    2 & Product & 7 \\
    \hline
    3 & Academy & 6 \\
    \hline
    4 & Academy - Blockchain & 7 \\
    \hline
    5 & Academy - Blockchain - Article & 8 \\
    \hline
    6 & Glossary & 9 \\
    \hline
    7 & Term of the Glossary & 9 \\
    \hline
    8 & Blog & 6 \\
    \hline
    9 & Blog article & 7 \\
    \hline
    10 & Support & 7 \\
    \hline
    & \textbf{Final grade} & \textbf{7/10} \\
    \hline
  \end{longtable}
\end{center}