\section{Conclusions and assessment}

\subsection{Conclusions}

\subsubsection{Positive aspects}

\paragraph{Support page}

This page is well organized:
\begin{itemize}
  \item There is a search bar that allows you to search for answers to 
  doubts concerning the products made by the company;

  \item A section that offers some FAQs, already divided by product;
  
  \item Specific sections that allow the user to reach the articles of the 
  \textit{Blog} or the \textit{Academy} and the various social network 
  links.
\end{itemize}
By doing so, the site suggests to the user:
\begin{itemize}
  \item Before finding the answer to the problem by searching among the 
  resources published on the site (articles from the \textit{Blog} or 
  the \textit{Academy});

  \item To ask for the help of the community through the various social 
  networks.
\end{itemize}


\paragraph{Search tools for the glossary page}

This page offers some great term search tools. If the user knows the term 
to search, he can use the search bar; instead if he partially remembers 
the term, he can:
\begin{itemize}
  \item Use the search bar, as the page shows all the results that 
  coincide with the characters entered in the search bar;

  \item The user can use the tool located to the right of the search bar: 
  it consists of an alphabet, in which each letter is a link that points 
  to a section in which all the terms starting with the selected letter 
  are illustrated.
\end{itemize}
These tools also allow a novice user of the cryptocurrency world to 
explore the glossary efficiently.

\subsubsection{Negative aspects}

\paragraph{How to open a ticket on the support page}

Despite the positive aspects described above, this page does not highlight 
the button that allows you to open a support ticket (the button can be 
located at the top right, to the left of the green button \textit{Login}, 
fig. \ref{fig:support-1}). This is a problem, as the user may be 
disoriented and not knowing how to contact support.

\paragraph{Lack of research tools for the Blog and for the Academy pages}

There are no search tools for either the \textit{Blog} or the 
\textit{Academy}. This represents a major disadvantage for the user, as:
\begin{itemize}
  \item The user must scroll through the various pages of the \textit{Blog} 
  or \textit{Academy} section in order to find the article of interest to 
  the user, or,

  \item The user has to search for the article via a search engine, which 
  will redirect it back to the website.
\end{itemize}
These tools are very useful for users (especially novice users) who need 
to find articles by means of keywords. With the addition of such tools, 
the usability of these pages would increase considerably.

\paragraph{The glossary page is not closely related to the academy page}

The glossary page should be easy to reach from an article or from the 
\textit{Academy} page itself. In the articles, there are not always links 
that redirect to the corresponding term in the glossary. Therefore, it 
would be useful to add a link, so that the user can consult the glossary 
at any time.

\paragraph{FAQ}

The FAQ is broken down by topic, which is a good thing, as if the user has 
been able to locate the problem, they can go directly to the specific 
section. However, there are three main problems:
\begin{itemize}
  \item These pages are difficult to reach: these pages can be reached via 
  the footer (fig. \ref{fig:footer});
  
  \item The various FAQs are not grouped into a single page: for a novice 
  user, it is useful to have a reference page that directs him to the 
  specific section (for example, \ textit {Taxation}). In the absence of 
  such a page, it is difficult for the user to orient himself and 
  therefore he could open a support ticket when the answer to his problem 
  could be present precisely in the FAQ;

  \item Absence of search tools: there are no tools that allow you to 
  search for FAQs starting from some terms entered by the user. This is a 
  major problem, as it would have facilitated usability for the user.
\end{itemize}

\subsection{Assessment}

Based on all the analyzes carried out, from the homepage to some internal 
pages, this site is able to provide the contents in a clear way. The 
sections within the pages are well organized, but lack the tools that 
would be essential for users, especially cryptocurrency newbies. So, the 
final grade that I think is adequate for the Young Platform website is 
\textbf{7/10}. The following table illustrates the grades assigned for 
each page.

\begin{center}
  \begin{longtable}{|c|c|c|}
    \hline
    & \textbf{Page} & \textbf{Grade} \\
    %\endhead
    \hline
    1 & Homepage & 8 \\
    \hline
    2 & Product & 7 \\
    \hline
    3 & Academy & 6 \\
    \hline
    4 & Academy - Blockchain & 7 \\
    \hline
    5 & Academy - Blockchain - Article & 8 \\
    \hline
    6 & Glossary & 9 \\
    \hline
    7 & Term of the Glossary & 9 \\
    \hline
    8 & Blog & 6 \\
    \hline
    9 & Blog article & 7 \\
    \hline
    10 & Support & 7 \\
    \hline
    & \textbf{Final grade} & \textbf{7/10} \\
    \hline
  \end{longtable}
\end{center}