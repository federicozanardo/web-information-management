\section{Introduction}

\subsection{Definition of usability}
L'\textit{usabilità} è un attributo di qualità che valuta quanto è facile 
utilizzare un prodotto da parte di un utente per il raggiungimento dei 
propri obiettivi.  Tale misura è differente rispetto  
all'\textit{accessibilità}, in quanto indica invece la capacità 
di un sito di fornire un'esperienza utente anche per gli utenti che 
presentano delle disabilità. Grazie allo standard ISO 9241-11:1998 è 
possibile fornire una definizione più precisa, in particolare, l'usabilità 
è una misura di:
\begin{itemize}
  \item \textbf{Efficacia}: indica la precisione e la completezza con cui 
  gli utenti raggiungono il loro obiettivo;
  \item \textbf{Efficienza}: indica le risorse spese in rapporto alla 
  precisione e alla completezza cui gli utenti raggiungono il loro 
  obiettivo;
  \item \textbf{Soddisfazione}: indica il grado di comfort con cui gli 
  utenti sono riusciti ad arrivare ai propri obiettivi attraverso 
  l'utilizzo del sistema.
\end{itemize}

In particolare, nel contesto dei siti web, l'accessibilità è una misura di 
qualità che specifica quanto un’interfaccia utente sia facilmente 
utilizzabile, ed è possibilibe identificare altre tre componenti:
\begin{itemize}
  \item \textbf{Apprendibilità}: indica quanto è facile per un utente che 
  utilizza per la prima volta un sito web arrivare compiere delle azioni 
  semplice (i.e., la navigazione delle pagine);
  \item \textbf{Memorabilità}: indica quanto è facile per un utente, che non 
  frequenta il sito web da molto tempo, capire come utilizzarlo;
  \item \textbf{Errori}: indica quanti e quanto gravi sono gli errori che 
  l'utente compie durante la navigazione nel sito web. Inoltre, indica quanto 
  tempo impiega l'utente per risolverli.
\end{itemize}

Tutte queste componenti permettono di capire che l'usabilità è un fattore 
decisivo per il successo di un sito web. Il maggior utilizzo dei siti web 
al giorno d'oggi, implica che l'interfaccia deve essere chiara ed intuitiva 
per permettere anche ad utenti con basse conoscenze tecnologiche di 
raggiungere i propri obiettivi. Quindi, se il sito non è in grado di 
illustrare esplicitamente e chiaramente il proprio intento, l'utente 
lascerà il sito e cercherà un altro servizio.