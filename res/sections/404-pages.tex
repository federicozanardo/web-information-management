\section{404 pages}

The site has not developed a specific page to handle the case in which the 
user requests a specific page that does not exist. You can find two 
different 404 pages: figure \ref{fig:404-academy} illustrates the result 
if a non-existent \textit{Academy} article is searched and figure 
\ref{fig:404-support} illustrates the result if non-existent content of the 
page \textit{Supporto} (\textit{Support}) is searched. These pages are very 
raw and do not provide useful information to the user. The use of the error 
code 404 can be omitted, as it represents a technicality that is not 
helpful to the user. The link to return to the homepage in figure 
\ref{fig:404-support} also represents a reference to be able to return to 
a familiar point for the user, however it also represents a disadvantage: 
if the user has reached this non-existent page via the 
\textit{deep linking}, making the user go back would mean leaving the 
site. In the figure \ref{fig:404-academy} it is possible to see that the 
404 page is not managed well: apart from the ironic phrase, all the texts 
of the menus, buttons and footer are not loaded. This can cause severe 
disorientation for the user. Among other things, being the section of 
learning articles, therefore frequented above all by novice users, it 
does not contribute to maintaining a good level of quality.

\begin{figure}[H]
  \centering
  \includegraphics[width=0.80\textwidth]{res/images/404-academy.png}
  \caption{Page 404 which occurs if an article from the \textit{Academy} 
  page is not found.}
  \label{fig:404-academy}
\end{figure}

\begin{figure}[H]
  \centering
  \includegraphics[width=0.80\textwidth]{res/images/404-support.png}
  \caption{Page 404 which appears if you search for content not present 
  in the page \textit{Supporto} (\textit{Support}).}
  \label{fig:404-support}
\end{figure}