\section{Preliminary analysis}

\subsection{Context}

\href{https://youngplatform.com}{YoungPlatform} è un exchange di 
criptovalute italiano. Uno dei principali obiettivi di questa azienda è 
quello di avvicinare coloro che non hanno mai sentito parlare di 
criptovalute o che non hanno le hanno mai utilizzate, al mondo delle 
criptovalute. Perciò hanno intuito che, invece di sviluppare direttamente 
l'exchange, era più utile educare gli utenti ed introdurli in questo 
mondo. Quindi, hanno deciso di sviluppare una piattaforma per 
l'apprendimento, che loro hanno chiamato \textit{Academy}. Pertanto, il 
sito web deve essere fruibile sopratutto da utenti che non hanno esperienza 
nel settore e che sono interessati ad imparare. 

\subsection{Website name}

La scelta del nome del sito web è determinante per fare in modo che sia 
facile da ricordare per gli utenti. Inoltre, la scelta del nome incide 
in media tra il 10\% ed il 20\% nell'usabilità e nella qualità del sito web. 
In questo caso, il nome del sito web è composto da due termini: 
\textit{young} e \textit{platform}. Il termine \textit{young} è stato 
inserito per evidenziare l'età molto giovane dei fondatori dell'azienda. 
Invece \textit{platform} è stato inserito per indicare che l'obiettivo 
dell'azienda è quello di creare una serie di prodotti, un 
\textit{ecosistema}, con un target diverso di utenti: prodotti per neofiti, 
prodotti per utenti con una media dimistichezza dell'argomento e prodotti 
per utenti esperti. I vantaggi dell'utilizzo di questo nome è che non 
utilizza delle parole inventate, ma è la composizione di due parole 
inglesi note. Per concludere, il nome non permette ad un neofita di 
intuire immediatamente lo scopo del sito web e quali sono i 
contenuti/prodotti che offre, invece, per un utente che ha conoscenza 
degli argomenti, tale nome rappresenta un punto di riferimento delle 
criptovalute nel contesto italiano.

\subsection{Search Engine Optimization}

Tramite il tool \textit{LightHouse}, il sito web ha ottenuto un punteggio 
elevato (92). In particolare, è possibile notare l'utilizzo del tag 
\verb|<meta name="viewport">|, in quanto ottimizza il sito per gli schermi 
dei dispositivi mobili di varie dimensioni. Inoltre, sono state aggiunte 
delle meta descrizioni. Tali descrizioni possono essere incluse nei 
risultati di ricerca, utili per riassumere il contenuto della pagina.

\subsection{Search Engine Results Page positioning}

Se si inseriscono dei termini di ricerca brevi e semplici in italiano, 
in quanto il target dell'azienda per il momento è il pubblico italiano, 
si può notare che il sito compare nei primi 10 risultati. L'azienda ha 
sviluppato un prodotto che permette di guadagnare delle criptovalute 
semplicemente camminando. Quindi, inserendo dei termini di ricerca 
strettamente relativi alle funzionalità di tale prodotto, si suppone che 
il sito compaia tra i primi 10 risultati. Tuttavia questo non è accaduto, 
in quanto il sito è stato trovato ad una posizione maggiore di 20. Di 
seguito, si illustrino alcuni risultati:
\begin{itemize}
  \item \verb|crytpo exchange|: 8
  \item \verb|crypto exchange italiano|: 1
  \item \verb|comprare bitcoin|: 5
  \item \verb|comprare ethereum|: 7
  \item \verb|comprare crypto|: 4
  \item \verb|comprare criptovalute|: 7
  \item \verb|vendere crypto|: 6
  \item \verb|vendere criptovalute|: 6
  \item \verb|cos'è bitcoin|: oltre i primi 20 risultati
  \item \verb|cos'è la blockchain|: oltre i primi 20 risultati
  \item \verb|guadagnare crypto|: oltre i primi 20 risultati
  \item \verb|guadagnare criptovalute|: oltre i primi 20 risultati
  \item \verb|guadagnare criptovalute gratis|: oltre i primi 20 risultati
  \item \verb|guadagnare criptovalute giocando|: oltre i primi 20 risultati
  \item \verb|guadagnare criptovalute camminando|: oltre i primi 20 
  risultati. Come terzo risultato vi è un articolo che parla del prodotto 
  che permette di guadagnare delle criptovalute camminando, in posizione  
  5 e 6 ci sono come risulati i riferimenti all'AppStore e al Google Play 
  Store dell'app di tale prodotto.
\end{itemize}

È possibile notare che ci sono diverse parole chiave molto importanti che 
non vengono sfruttate per portare il sito web nelle prime posizioni della 
pagina dei risultati del motore di ricerca. È più probabile che tale sito 
venga trovato tramite articoli pubblicati da altri siti web.